\startenvironment env_NotesDeCours

  % Couleurs

  \usecolors[crayola]

  \definecolor[CouleurTraits][black]%CosmicCobalt




  % Macros générales (non mathématiques)

  \define[1]\defterme{\begingroup\em #1\endgroup}


  % Layout

  \mainlanguage[fr]
  \language[fr]
  \setcharacterspacing[frenchpunctuation]

  \setuplayout[
    backspace=20mm,
    width=fit,
    topspace=10mm,
    height=fit]

  \setuppagenumbering[
    %alternative=doublesided,
    state=start,
    location={header,right}]

  \setupbodyfont[gentium, 12pt]


  % Titres de section

  \setupthinrules[
    rulethickness=2bp,
    color=CouleurTraits]

  \setuphead[chapter][
    alternative=middle,
    after=\thinrule,
    style=\bfd]

  \setuphead[section][style=\bfb]



  % Environnements mathématiques

  % Démonstration

  \definenarrower[demCORPS][
    default={left,right},
    left=1cm,
    right=.5cm]

  \define\startdem{
    \begingroup
    \setuptextrules[
      rulethickness=.5bp,
      rulecolor=CouleurTraits,
      inbetween=\tf,
      style=\it,
      bodyfont=small]
    \startdemCORPS
    \starttextrule{Démonstration}
    \blank[halfline,samepage]}

  \define\stopdem{
    \blank[halfline,samepage]
    \stoptextrule
    \stopdemCORPS
    \endgroup
    \blank[2*line]}

  


  % Style théorème

  \definetextbackground[lisereGauche][
    mp=lisereGaucheMP,
    location=paragraph,
    before={\blank[quarterline]},
    leftoffset=1em,
    topoffset=0ex,
    bottomoffset=0ex,
    rightoffset=0em,
    rulethickness=1bp,
    framecolor=CouleurTraits]

  \startuseMPgraphic{lisereGaucheMP}
    begingroup;
      for i=1 upto nofmultipars :
        draw leftboundary multipars[i]
        boxlineoptions withpen pensquare scaled boxlinewidth ;
      endfor ;
    endgroup;
  \stopuseMPgraphic

  \setupenumeration[
    title=yes,
    alternative=hanging,
    titlestyle=\bf,
    style=\it,
    counter=thm,
    width=fit,
    titleleft={---\enspace},
    titleright=,
    after={\blank[5*quarterline]},
    way=bysection,
    prefixsegments=chapter:section,
    prefix=yes]


  % Théorème

  \defineenumeration[theoremeENUM]   [text={Théorème}]
  \defineenumeration[propositionENUM][text={Proposition}]
  \defineenumeration[lemmeENUM]      [text={Lemme}]
  \defineenumeration[corollaireENUM] [text={Corollaire}]

  \define\starttheoreme
    {\dosingleempty\dostarttheoreme}
  \define\startproposition
    {\dosingleempty\dostartproposition}
  \define\startlemme
    {\dosingleempty\dostartlemme}
  \define\startcorollaire
    {\dosingleempty\dostartcorollaire}

  \def\dostarttheoreme[#1]{
      \starttheoremeENUM[title=#1]
      \page[no]
      \startlisereGauche}
  \def\dostartproposition[#1]{
      \startpropositionENUM[title=#1]
      \page[no]
      \startlisereGauche}
  \def\dostartlemme[#1]{
      \startlemmeENUM[title=#1]
      \page[no]
      \startlisereGauche}
  \def\dostartcorollaire[#1]{
      \startcorollaireENUM[title=#1]
      \page[no]
      \startlisereGauche}

  \define\stoptheoreme{
    \stoplisereGauche
    \stoptheoremeENUM}
  \define\stopproposition{
    \stoplisereGauche
    \stoppropositionENUM}
  \define\stoplemme{
    \stoplisereGauche
    \stoplemmeENUM}
  \define\stopcorollaire{
    \stoplisereGauche
    \stopcorollaireENUM}


  % Définition

  \defineenumeration[definitionENUM][
    text={Définition},
    style=\tf]
    

  \define\startdefinition
    {\dosingleempty\dostartdefinition}

  \def\dostartdefinition[#1]{
    \startdefinitionENUM[title=#1]
    \page[no]
    \startlisereGauche}
    
  \define\stopdefinition{
    \stoplisereGauche
    \stopdefinitionENUM}
  

  % Discussion

  \defineenumeration[discussionENUM][
    text={},
    stopper={.},
    style=\tf,
    titlestyle=\it,
    titleleft=,
    titleright={.\enspace---}]
    

  \define\startdiscussion
    {\dosingleempty\dostartdiscussion}

  \def\dostartdiscussion[#1]{
    \startdiscussionENUM[title=#1]}
    
  \define\stopdiscussion{
    \stopdiscussionENUM}


\stopenvironment
