\startenvironment env_NotesDeCours

  % Couleurs

  

  \usecolors[crayola]

  \definecolor[CouleurTraits][black]




  % Macros générales (non mathématiques)

  \define[1]\defterme{\begingroup\em #1\endgroup}


  % Layout

  \setupbodyfont[xits,12pt]

  \mainlanguage[fr]
  \language[fr]
  \setcharacterspacing[frenchpunctuation]

  \setuplayout[
    backspace=20mm,
    width=fit,
    topspace=10mm,
    height=fit]

  \setuppagenumbering[
    %alternative=doublesided,
    state=start,
    location={footer,middle}]


  \setupwhitespace[big]


  % Titres de section

  \setupthinrules[
    rulethickness=2bp,
    color=CouleurTraits]

  \setuphead[chapter][
    alternative=middle,
    after=\thinrule,
    style=\bfd]

  \setuphead[section][style=\bfb]



  %%%%%%%%%%%%%%%%%%

  % Listes structurées

  


  \defineconversion[puces][---\enspace, ---\enspace]
  
  \setupitemize[each][puces, fit]

  \setupitemize[
    leftmargin=1.5em,
    stopper=]


  \define\sansItalique{%
    \doif{\fontalternative}{it}{\tf}%
    \doif{\fontalternative}{bi}{\bf}%
  }


  
  \def\traiterArgsListe[#1]{%
    \doifinset{compact}{#1}{%
      \setupitemize[nowhite] % joinedup ?
    }%
    \doifinset{fin}{#1}{%
      \setupitemize[after={\blank[quarterline]}]%
    }%
  }

  \define\startlisteEnum
    {\dosingleempty\dostartlisteEnum}
  \define\startlisteNum
    {\dosingleempty\dostartlisteNum}
  \define\startlisteAlph
    {\dosingleempty\dostartlisteAlph}
  \define\startlisteRom
    {\dosingleempty\dostartlisteRom}


  \def\dostartlisteEnum[#1]{%
    \begingroup%
    \traiterArgsListe[#1]%
    \startitemize[puces]%
  }

  \def\dostartlisteNum[#1]{%
    \begingroup%
    \traiterArgsListe[#1]%
    \startitemize[n, fit][
      left=(,
      right=)\enspace,
      alignsymbol=yes,
      itemalign=left]%
  }

  \def\dostartlisteAlph[#1]{%
    \begingroup%
    \traiterArgsListe[#1]%
    \startitemize[a, fit][
      left=(,
      right=)\enspace,
      alignsymbol=yes,
      itemalign=left]%
  }

  \def\dostartlisteRom[#1]{%
    \begingroup%
    \traiterArgsListe[#1]%
    \startitemize[r, fit][
      left=(,
      right=)\enspace,
      alignsymbol=yes,
      itemalign=left]%
  }


  \define\stoplisteEnum
    {\stopitemize\endgroup}
  \define\stoplisteNum
    {\stopitemize\endgroup}
  \define\stoplisteAlph
    {\stopitemize\endgroup}
  \define\stoplisteRom
    {\stopitemize\endgroup}


  %%%%%%%%%%%%%%%%%%

  % Environnements mathématiques

  % Démonstration

  \definenarrower[demCORPS][
    default={left,right},
    left=1cm,
    right=.5cm]

  \define\startdem{
    \begingroup
    \setuptextrules[
      rulethickness=.5bp,
      rulecolor=CouleurTraits,
      inbetween=\tf,
      style=\it,
      bodyfont=small]
    \startdemCORPS
    \starttextrule{Démonstration}
    \page[no]\blank[halfline,samepage]}

  \define\stopdem{
    \blank[halfline,samepage]\page[no]
    \stoptextrule
    \stopdemCORPS
    \endgroup
    \blank[2*line]}

  


  % Style théorème

  \definetextbackground[lisereGauche][
    mp=lisereGaucheMP,
    location=paragraph,
    before={\blank[quarterline]},
    leftoffset=1em,
    topoffset=0ex,
    bottomoffset=0ex,
    rightoffset=0em,
    rulethickness=1bp,
    framecolor=CouleurTraits]

  \startuseMPgraphic{lisereGaucheMP}
    begingroup;
      for i=1 upto nofmultipars :
        draw leftboundary multipars[i]
        boxlineoptions withpen pensquare scaled boxlinewidth ;
      endfor ;
    endgroup;
  \stopuseMPgraphic

  \setupenumeration[
    title=yes,
    alternative=hanging,
    titlestyle=\bf,
    style=\it,
    counter=thm,
    width=fit,
    titleleft={---\enspace},
    titleright=,
    after={\blank[5*quarterline]},
    way=bysection,
    prefixsegments=chapter:section,
    prefix=yes]


  % Théorème

  \def\commencerThm[#1][#2][#3]{%
    \csname start#1ENUM\endcsname[reference={#2}, title={#3}]%
    \page[no]%
    \startlisereGauche%
  }


  \defineenumeration[theoremeENUM]   [text={Théorème}]
  \defineenumeration[propositionENUM][text={Proposition}]
  \defineenumeration[lemmeENUM]      [text={Lemme}]
  \defineenumeration[corollaireENUM] [text={Corollaire}]

  \define\starttheoreme
    {\dodoubleempty\dostarttheoreme}
  \define\startproposition
    {\dodoubleempty\dostartproposition}
  \define\startlemme
    {\dodoubleempty\dostartlemme}
  \define\startcorollaire
    {\dodoubleempty\dostartcorollaire}

  \def\dostarttheoreme[#1][#2]{%
    \commencerThm[theoreme][thm:#1][#2]}
  \def\dostartproposition[#1][#2]{%
    \commencerThm[proposition][prop:#1][#2]}
  \def\dostartlemme[#1][#2]{%
    \commencerThm[lemme][lem:#1][#2]}
  \def\dostartcorollaire[#1][#2]{%
    \commencerThm[corollaire][cor:#1][#2]}

  \define\stoptheoreme{
    \stoplisereGauche
    \stoptheoremeENUM}
  \define\stopproposition{
    \stoplisereGauche
    \stoppropositionENUM}
  \define\stoplemme{
    \stoplisereGauche
    \stoplemmeENUM}
  \define\stopcorollaire{
    \stoplisereGauche
    \stopcorollaireENUM}


  \definereferenceformat
    [inthm] [text={théorème }]
  \definereferenceformat
    [inprop][text={proposition }]
  \definereferenceformat
    [inlem] [text={lemme }]
  \definereferenceformat
    [incor] [text={corollaire }]

  \def\cref[#1:#2]{%
    \expandafter\ifx\csname in#1\endcsname\relax
      \writestatus{warning}{referenceformat in#1 not defined}%
      \in[#1:#2]%
    \else
      \csname in#1\endcsname[#1:#2]%
    \fi
  }


  % Définition

  \defineenumeration[definitionENUM][
    text={Définition},
    style=\tf]
    

  \define\startdefinition
    {\dosingleempty\dostartdefinition}

  \def\dostartdefinition[#1]{
    \startdefinitionENUM[title=#1]
    \page[no]
    \startlisereGauche}
    
  \define\stopdefinition{
    \stoplisereGauche
    \stopdefinitionENUM}


  % Assertion
  
  \defineenumeration[assertion][
    counter=assertion,
    text={Assertion},
    prefix=no]

  \setupcounter[assertion][way=bythm]



  % Style = remarque

  \setupenumeration[style=\tf]
  
  \defineenumeration[remarque][text={Remarque}]
  \defineenumeration[exemple][text={Exemple}]
  


  % Discussion

  \defineenumeration[discussionENUM][
    text={},
    stopper={.},
    style=\tf,
    titlestyle=\it,
    titleleft=,
    titleright={.\enspace---}]
  
  \setupenumeration[subdiscussionENUM][
    way=bythm,
    prefixsegments=chapter:*]

  \define\startdiscussion
    {\dosingleempty\dostartdiscussion}
  \define\startsousdiscussion
    {\dosingleempty\dostartsousdiscussion}

  \def\dostartdiscussion[#1]{
    \startdiscussionENUM[title=#1]}
  \def\dostartsousdiscussion[#1]{
    \startsubdiscussionENUM[title=#1]}
    
  \define\stopdiscussion{
    \stopdiscussionENUM}
  \define\stopsousdiscussion{
    \stopsubdiscussionENUM}


\stopenvironment
