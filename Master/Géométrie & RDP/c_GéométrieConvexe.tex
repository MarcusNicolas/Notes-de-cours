\startcomponent c_GéométrieConvexe
\product prd_GéométrieRDP

\startchapter[title={Un peu de géométrie convexe}]

  %%%
  \startsection[title={Faces d'un ensemble convexe}]

    \startdiscussion[Convexité]
      On rappelle, étant donné un espace affine $A$ sur $\reels$, qu'une partie $C \subseteq A$ est dite \defterme{convexe} lorsque le segment $\segment{x}{y}$ est contenu dans $C$ pour tout couple $\p{x, y}$ de points de $C$. La \defterme{dimension} de $C$, notée $\dim{C}$, est alors la dimension de son enveloppe affine $\aff{C}$. L'intérieur et la frontière de $C$ sont toujours considérés relativement à l'espace topologique $\aff{C}$.
    \stopdiscussion


    \startremarque
      Un convexe non vide est d'intérieur non vide.
    \stopremarque



    \starttheoreme[caratheodory][Carathéodory]
      Tout point de l'enveloppe convexe d'une partie $A$ d'un espace affine de dimension $d$ est barycentre de $d+1$ points de $A$.
    \stoptheoreme


    \startdem
      Étant donné $y \in \conv{A}$, il existe un entier $r$ minimal, des réels positifs $\lambda_i$ et des points $x_i$ de $A$ vérifiant:
      \startformula
        y = \sum_{i=0}^r \lambda_i x_i
        \quad\text{et}\quad
        \sum_{i=0}^r \lambda_i = 1
      \stopformula
      Si $r \geq d+1$, alors la famille $x_0, \dots, x_r$ n'est pas affinement indépendante. Autrement dit, il existe des réels $\mu_0, \dots, \mu_r$ non tous nuls tels que:
      \startformula
        \sum_{i=0}^r \mu_i x_i = 0
        \quad\text{et}\quad
        \sum_{i=0}^r \mu_i = 0
      \stopformula
      On peut supposer avoir $\mu_r > 0$ et $\lambda_r / \mu_r < \lambda_i / \mu_i$ pour tout $i$ tel que $\mu_i > 0$. Alors:
      \startformula
        x_j = -\sum_{i=0}^{r-1} \frac{\mu_i}{\mu_r} x_i
        \quad\text{puis}\quad
        y = \sum_{i=0}^{r-1}
          \p[*]{\lambda_i - \frac{\lambda_r}{\mu_r} \mu_i} x_i
      \stopformula
      et on vérifie immédiatement que:
      \startformula
        \sum_{i=0}^{r-1} \p[*]{\lambda_i - \frac{\lambda_r}{\mu_r} \mu_i} = 1
        \qquad\text{et}\qquad
        \pourTout{i < r} \lambda_i - \frac{\lambda_r}{\mu_r} \mu_i \geq 0
      \stopformula
      C'est absurde car $r$ est supposé minimal, donc $r \leq d$.
    \stopdem


    On travaille implicitement dans le reste de chapitre dans un espace affine euclidien ambiant $E$.

    \startdefinition
      Soit $K$ un convexe fermé. Une \defterme{face exposée} de $K$ est une partie de $K$ vide ou de la forme $H \cap K$ avec $H$ hyperplan d'appui de $K$ dans $\aff{K}$.
      
      L'ensemble des faces exposées de $K$ est noté $\facesExp\p{K}$.\\
      Ses éléments maximaux pour la relation d'inclusion forment l'ensemble $\facesMax\p{K}$.
    \stopdefinition


    \startdiscussion[Premières propriétés]
      Si $F = H \cap K$ est face exposée non vide d'un convexe fermé $K$, alors $\aff{F} \subseteq H$ puis $\aff{F} \cap K \subseteq H \cap K = F$. L'inclusion réciproque étant évidente, il vient que:
      \startformula
        F = \aff{F} \cap K
      \stopformula
      et c'est vrai également dans le cas où $F$ est vide.
      
      En particulier, si $F \subseteq F'$ sont deux faces exposées de $K$ de même dimension, alors $F = F'$. Cela assure par exemple que l'ensemble $\facesExp\p{K}$ est inductif.
    \stopdiscussion


    \startproposition
      Étant donné un convexe fermé $K$:
      \startlisteNum[fin]
        \startitem
          Les éléments de $\facesExp\p{K}$ sont des parties convexes fermées de $\frontiere{K}$.
        \stopitem
        \startitem
          L'ensemble $\facesExp\p{K}$ est stable par intersections finies non vides.
        \stopitem
      \stoplisteNum
    \stopproposition

    \startdem
      \startlisteNum
        \startitem
          Clair.
        \stopitem

        \startitem
          Soient $F_1, \dots, F_m \in \facesExp\p{K}$. On note:
          \startformula
            F = \bigcap_{i=1}^m F_i
          \stopformula
          Si $F$ est vide alors il n'y a rien à faire. Sinon on fixe $a \in F$. Il existe alors pour tout $i$ un vecteur non nul $u_i$ tel que, si $H_i$ désigne l'hyperplan d'équation $\prodScal{x-a}{u_i} = 0$:
          \startformula
            F_i = H_i \cap K
            \quad\text{et}\quad
            \pourTout{x \in K} \prodScal{x-a}{u_i} \leq 0
          \stopformula
          Le vecteur $u = \sum_i u_i$ est non nul, car si $b \in \interieur{K}$ alors $\prodScal{b-a}{u} < 0$.\\
          Finalement, si $H$ désigne l'hyperplan d'équation $\prodScal{x}{u} = 0$:
          \startformula
            F = H \cap K
            \quad\text{et}\quad
            \pourTout{x \in K} \prodScal{x}{u} \leq 0
          \stopformula
          et $H$ est un hyperplan d'appui de $K$ en $a$, d'où $F \in \facesExp\p{K}$.
        \stopitem
      \stoplisteNum
    \stopdem


    On fixe dans la suite un convexe fermé $K$.


    \startproposition[frontiere_reunion_faces]
      Si $K$ n'est pas réduit à un point, alors $\frontiere{K} = \bigcup \facesMax\p{K}$.
    \stopproposition

    \startdem
      Si $K$ est vide, il n'y a rien à faire. Sinon, le théorème de séparation des convexes assure que tout point de la frontière de $K$ appartient à une face exposée, et le lemme de Zorn permet de conclure.
    \stopdem


    \startremarque
      Si $C' \in \facesExp\p{C}$ et $C'' \in \facesExp\p{C'}$, on n'a pas nécessairement $C'' \in \facesExp\p{C}$. REGARDER LE CONVEXE SUIVANT DANS $\reels^2$, POINTS EXTREMAUX $\neq$ POINTS EXPOSÉS. % TODO
      \placefigure{}{
        \useMPgraphic{mon_graphique}
      }
    \stopremarque


    La remarque précédente justifie la définition suivante :

    \startdefinition
      Une \defterme{face} de $K$ est une partie $F$ de $K$ telle qu'il existe une suite de convexes fermés:
      \startformula
        K = F_0 \supseteq \dots \supseteq F_k = F
      \stopformula
      telle que $F_i \in \facesExp\p{F_{i-1}}$ pour $1 \leq i \leq k$.

      On désigne par $\faces\p{K}$ l'ensemble des faces de $K$.
    \stopdefinition


    \startremarque
      La suite vide montre que $K$ est une face de $K$. L'ensemble des faces propres maximales de $K$ est exactement $\facesMax\p{K}$.
    \stopremarque


    \startproposition[transitivite_faces][Transitivité]
      Si $F$ est une face de $K$, alors les faces de $F$ sont les faces de $K$ contenues dans $F$.
    \stopproposition

    \startdem
      Il suffit de traiter le cas où $F$ est une face exposée de $K$.\\
      Étant donnée une face $F'$ de $K$ contenue dans $F$, on considère une suite $K = F'_0 \supseteq \dots \supseteq F'_k = F'$ telle que $F'_i \in \facesExp\p{F'_{i-1}}$ pour tout $i$. Si $F'_i$ s'écrit $F'_{i-1} \cap H$ avec $H$ hyperplan d'appui $F'_{i-1}$, alors :
      \startformula
        F \cap F'_i = \p{F \cap F'_{i-1}} \cap H
      \stopformula
      et $H$ est un hyperplan d'appui de $F \cap F'_{i-1}$. Ainsi $F \cap F'_i \in \facesExp\p{F \cap F'_{i-1}}$, et c'est évidemment aussi le cas si $F'_i$ est vide. La suite :
      \startformula
        F = F \cap F'_0 \supseteq \dots \supseteq F \cap F'_k = F'
      \stopformula
      montre finalement que $F'$ est une face de $F$.
    \stopdem


    \startproposition[ptExt_faces]
      Si $F$ est une face de $K$, alors $\ptExt{F} = F \cap \ptExt{K}$.
    \stopproposition

    \startdem
      On peut supposer que $F$ est une face exposée non vide, et on écrit dans ce cas $F = H \cap K$ avec $H$ hyperplan d'appui de $K$. Étant donné $a \in \ptExt{F}$, il existe un vecteur $u$ non nul tel que :
      \startformula
        H = \comprehension{x}{\prodScal{x-a}{u}=0}
        \quad\text{et}\quad
        \pourTout{x \in K} \prodScal{x-a}{u} \leq 0
      \stopformula
      Supposons qu'il existe $0 < \lambda < 1$ et deux points distincts $x$ et $y$ de $K$ tels que $a = \lambda x + \p{1-\lambda} y$. Alors:
      \startformula
        \lambda \prodScal{x-a}{u} + \p{1-\lambda} \prodScal{y-a}{u} = 0
      \stopformula
      ce qui impose que $x$ et $y$ appartiennent à $F$. C'est absurde, d'où:
      \startformula
        \ptExt{F} \subseteq F \cap \ptExt{K}
      \stopformula
      et l'inclusion réciproque est immédiate.
    \stopdem


    \startcorollaire[ptExt_faces_singletons]
      Les points extrémaux de $K$ sont exactement les faces singletons.
    \stopcorollaire

    \startdem
      Si $\classe{x}$ est une face singleton de $K$, la \cref[prop:ptExt_faces] assure que :
      \startformula \startalign
        \NC \classe{x} \cap \ptExt{K} \NC = \ptExt{\classe{x}} \NR
        \NC \NC = \classe{x} \NR
      \stopalign \stopformula
      Autrement dit, $x$ est un point extrémal de $K$.

      Réciproquement, considérons $x$ un point extrémal de $K$. Si $K$ est réduit à un point, alors il n'y a rien à faire. Sinon il existe, en vertu de la \cref[prop:frontiere_reunion_faces], une face propre maximale $F$ de $K$ qui contient $x$. Le point $x$ est en particulier extrémal dans $F$ et $\dim{F} < \dim{K}$. En réitérant ce processus, on finit par montrer que $x$ appartient à une face de $K$ de dimension nulle, ce qui achève la preuve.
    \stopdem


  \stopsection


  %%%
  \startsection[title={Polytopes convexes}]

    \startdefinition
      Un \defterme{polytope convexe} est un convexe non vide qui est intersection finie de demi-espaces fermés dans son enveloppe affine.\\
      Les $0$-faces et $1$-faces d'un polytope sont respectivement appelés \defterme{sommets} et \defterme{arêtes}.
    \stopdefinition


    % FIXME : quand j'utilise les résultats de cette discussion, je ne vérifie
    % pas toujours l'hypothèse sur la dimension (mais si K est un point, tout
    % est évidemment trivial)
    \startdiscussion
      Étant donné un polytope convexe $X$ de dimension non nulle, on fixe une famille minimale de demi-espaces fermés $D_1, \dots, D_r$ de $\aff{X}$ vérifiant:
      \startformula
        X = \bigcap_{i=1}^r D_i
      \stopformula
      En fixant $a \in \interieur{X}$, on peut écrire pour tout $1 \leq i \leq r$:
      \startformula
        D_i = \comprehension{x}{\prodScal{x-a}{u_i} \leq \theta_i}
        \quad\text{et}\quad
        F_i = \frontiere{D_i} \cap X
      \stopformula
      où les $u_i$ sont des vecteurs non nuls et les $\theta_i$ des réels strictement positifs.


      \startsousdiscussion
        Étant donné un indice $1 \leq i \leq r$, on pose:
        \startformula
          X_i = \bigcap_{j \neq i} D_j
        \stopformula
        La minimalité de la famille $D_1, \dots, D_r$ implique $X \subset X_i$.\\
        Si $b \in X_i - X$, alors $\prodScal{b-a}{u_i} > \theta_i$ et il existe $c \in \intervalle{a}{b} \subseteq X_i$ tel que:
        \startformula
          \prodScal{c}{u_i} = \theta_i
        \stopformula
        Alors $c \in \frontiere{D_i} \cap X_i = F_i$, et $\frontiere{D_i}$ est un hyperplan d'appui de $X$ en $c$.\\
        En particulier $F_i$ est une face exposée de $X$, et $F_i \subseteq \frontiere{X}$ d'après la \cref[prop:frontiere_reunion_faces].

        Maintenant, si $x \in X - \bigcup_i F_i$, alors $\prodScal{x-a}{u_i} < \theta_i$ pour $1 \leq i \leq r$ et $X$ est voisinage de $x$. On en déduit $X - \bigcup_i F_i \subseteq \interieur{X}$, d'où $\frontiere{X} = X - \interieur{X} \subseteq \bigcup_i F_i$.
        
        Finalement:
        \startformula
          \frontiere{X} = \bigcup_{i=1}^r F_i
        \stopformula
      \stopsousdiscussion


      \startsousdiscussion
        Si $F \in \facesExp\p{X}$, alors d'après ce qui précède:
        \startformula
          F = F \cap \frontiere{X} = \bigcup_{i=1}^r \p{F \cap F_i}
        \stopformula
        Si $\lambda$ désigne la mesure de Lebesgue sur $\aff{X}$, alors :
        \startformula
          0 < \lambda\p{F} \leq \sum_{i=1}^r \lambda\p{F \cap F_i}
        \stopformula
        et il existe un indice $1 \leq i \leq r$ tel que $\lambda\p{F \cap F_i} > 0$. En particulier $F$ et $F \cap F_i$ sont de même dimension, puis $F \cap F_i = F$. Autrement dit $F \subseteq F_i$, et $\facesMax\p{X} \subseteq \classe{F_1, \dots, F_r}$.

        Réciproquement, supposons qu'il existe deux indices $i$ et $j$ tels que $F_i \subseteq F_j$.\\
        Si $b \in X_i - X$, alors $\prodScal{b-a}{u_i} > \theta_i$ et il existe $c \in \intervalle{a}{b} \subseteq X_i$ tel que $\prodScal{c-a}{u_i} = \theta_i$. En particulier, $c \in F_i$ et l'hypothèse assure que $\prodScal{c-a}{u_j} = \theta_j$, or $\prodScal{b-a}{u_j} \leq \theta_j$. C'est absurde, donc les $F_i$ sont maximales parmi les faces exposées de $X$.

        Ainsi $\facesMax\p{X} = \classe{F_1, \dots, F_r}$.
      \stopsousdiscussion


      \startsousdiscussion
        Pour $1 \leq i \leq r$, on a:
        \startformula \startalign
          \NC F_i \NC = \frontiere{D_i} \cap X_i \NR
          \NC \NC = \bigcap_{i \neq j} \p{\frontiere{D_i} \cap D_j} \NR
        \stopalign \stopformula
        or $\frontiere{D_i} \cap D_j$ est soit plein soit un demi-espace fermé dans $\frontiere{D_i}$. En particulier $F_i$ est un polytope convexe, et chaque $F \in \facesMax\p{F_i}$ est d'après ce qui précède de la forme:
        \startformula \startalign
          \NC F \NC = \frontiere\p{\frontiere{D_i} \cap D_j} \cap F_i \NR
          \NC \NC = \frontiere{D_i} \cap \frontiere{D_j} \cap F_i \NR
          \NC \NC = F_i \cap F_j
        \stopalign \stopformula
        pour un certain $j \neq i$ tel que $\frontiere{D_i} \cap D_j \subset \frontiere{D_i}$.

        Ainsi chaque élément de $\facesMax\p{F_i}$ est intersection de deux éléments de $\facesMax\p{X}$.
      \stopsousdiscussion
    \stopdiscussion


    \startproposition[intersections_facesMax_polytope]
      Une face de $X$ est intersection finie d'éléments de $\facesMax\p{X}$ dans $X$.
    \stopproposition

    \startdem
      On raisonne par récurrence sur la dimension $d$ de $X$:
      \startlisteEnum
        \startitem
          si $X$ est réduit à un point, alors l'énoncé est vide.
        \stopitem

        \startitem
          si $d \geq 1$, on se donne une face $F$ de $X$. Si $F = X$, alors l'intersection vide convient. Sinon il existe une face maximale propre $G$ de $X$ contenant $F$. Alors $F$ est une face de $G$ en vertu de la \cref[prop:transitivite_faces], et $F$ est intersection finie dans $G$ d'éléments de $\facesMax\p{G}$ d'après l'hypothèse de récurrence. La discussion qui précède permet de conclure.
        \stopitem
      \stoplisteEnum
    \stopdem


    On déduit immédiatement de ce qui précède:

    \starttheoreme[faces_polytope]
      Si $X$ est un polytope convexe, alors les faces exposées de $X$ sont exactement ses faces propres et sont en nombre fini.
    \stoptheoreme



    \startdiscussion
      % TODO : Digression, organiser ce qui vient après
    \stopdiscussion


    \startproposition[faces_polytope_espaceAmbiant]
      Soit $X$ un polytope convexe.\\
      Si $D_1, \dots, D_r$ sont des demi-espaces fermés de l'espace $E$ ambiant tels que:
      \startformula
        X = \bigcap_{i=1}^r D_i
      \stopformula
      alors, en notant $F_i = \frontiere{D_i} \cap X$ pour tout $1 \leq i \leq r$, les faces de $X$ sont alors exactement les intersections dans $X$ d'un nombre fini de $F_i$.
    \stopproposition

    \startdem
      On a:
      \startformula
        X = \bigcap_{i=1}^r \p{D_i \cap \aff{X}}
      \stopformula
      or $\Delta_i = D_i \cap \aff{X}$ est soit plein soit un demi-espace fermé dans $\aff{X}$ pour tout $i$. Or, pour $1 \leq i \leq r$, on vérifie que :
      \startformula
        F_i = \frontiere{\Delta_i} \cap X
      \stopformula
      En particulier, les $F_i$ sont des faces exposées (éventuellement vides) de $X$.

      Réciproquement, on extrait de la famille $\Delta_1, \dots, \Delta_r$ une famille minimale délimitant $X$ que l'on suppose, quitte à réarranger les $D_i$, de la forme $\Delta_1, \dots, \Delta_s$. Mais alors $\facesMax\p{X} = \classe{F_1, \dots, F_s}$ et toute face de $X$ est intersection dans $X$ d'un nombre fini de $F_i$ en vertu de la \cref[prop:intersections_facesMax_polytope].
    \stopdem


    \startcorollaire[faces_intersection_polytopes]
      Si $X_1$ et $X_2$ sont deux polytopes convexes, alors:
      \startformula
        \faces\p{X_1 \cap X_2} = \comprehension{F_1 \cap F_2}{\p{F_1, F_2} \in \faces\p{X_1} \times \faces\p{X_2}}
      \stopformula
    \stopcorollaire

    \startdem
      On peut écrire dans $E$:
      \startformula
        X_1 = \bigcap_{i=1}^r D_i
        \quad\text{et}\quad
        X_2 = \bigcap_{i=r+1}^{r+s} D_i
      \stopformula
      où les $D_i$ sont des demi-espaces fermés. Alors:
      \startformula
        X_1 \cap X_2 = \bigcap_{i=1}^{r+s} D_i
      \stopformula
      et la \cref[prop:faces_polytope_espaceAmbiant] permet de conclure.
    \stopdem


  \stopsection


  %%% TODO : réorganiser cette section ?
  \startsection[title={Cellules linéaires}]

    \startdefinition
      Une \defterme{cellule linéaire} est l'enveloppe convexe d'un nombre fini de points.
    \stopdefinition


    \startdiscussion[Diamètre et corpulence]
      On appelle \defterme{corpulence} d'une cellule $K = \cellule{a_0, \dots, a_k}$ le rapport:
      \startformula
        \corp\p{K} = \frac{r\p{K}}{d\p{K}}
      \stopformula
      où $r\p{K}$ est la distance séparant l'isobarycentre de $K$ à son bord et $d\p{K}$ est le diamètre de $K$ (qui est par convexité le maximum des $\abs{a_i - a_j}$).
    \stopdiscussion


    \startlemme[faces_conv_ptExt]
      Si $K$ est une cellule, alors tout point d'une $k$-face de $K$ appartient à l'enveloppe convexe de $k+1$ points extrémaux de $K$.
    \stoplemme

    \startdem
      Soient $K$ une cellule et $F$ une éventuelle $k$-face de $K$. Le théorème de Krein--Milman et la \cref[prop:ptExt_faces] assurent que $F$ est l'enveloppe convexe des points extrémaux de $K$ qui appartiennent à $F$. Le \cref[thm:caratheodory] appliqué dans l'espace $\aff{F}$ permet de conclure.
    \stopdem


    Le résultat suivant est fondamental:

    \starttheoreme
      Les cellules sont exactement les polytopes bornés.
    \stoptheoreme

    \startdem
      % TODO : réorganiser la preuve
      Soit $K$ une cellule, que l'on suppose sans perte de généralité de dimension $d \geq 1$. L'ensemble $L$ des combinaisons affines d'au plus $d-1$ points extrémaux de $K$ contient d'après le \cref[lem:faces_conv_ptExt] les faces de $K$ de dimension inférieure à $d-2$. Étant donné $b \nin K$, on définit :
      \startformula
        C = \comprehension
          {\lambda x + \p{1-\lambda} b}
          {\lambda \geq 0 \text{ et } x \in L}
      \stopformula
      L'ensemble $C$ est contenu dans un nombre fini d'espaces affines de dimension inférieure à $d-1$, donc est négligeable pour la mesure de Lebesgue dans $\aff{K}$. En particulier, il existe $a \in \p{\interieur{K}} \setminus C$ puis, par compacité de $K$, un $0 < \lambda < 1$ minimal tel que le point $c = \lambda a + \p{1-\lambda} b$ appartienne à $K$. Alors $c \in \frontiere{K}$, et il existe d'après la \cref[prop:frontiere_reunion_faces] une face propre $F$ de $K$ contenant $c$, nécessairement de dimension $d-1$. Si l'hyperplan $\aff{F}$ est d'équation $\prodScal{x-a}{u} = \theta$, alors $b-a$ n'est pas orthogonal à $u$ et $\aff{F}$ sépare $K$ du point $b$. On en déduit que $K$ est l'intersection des demi-espaces fermés déterminés par les $\p{d-1}$-faces de $K$, qui sont en nombre fini d'après le \cref[lem:faces_conv_ptExt].

      Réciproquement, si $K$ est un polytope borné, alors il possède un nombre fini de points extrémaux d'après le \cref[cor:ptExt_faces_singletons] et le \cref[thm:faces_polytope]. Le théorème de Krein--Milman permet de conclure.
    \stopdem


    \startcorollaire[intersection_cellules]
      L'intersection de deux cellules est encore une cellule.
    \stopcorollaire

  \stopsection


  %%%
  \startsection[title={Simplexes et complexes simpliciaux}]

    \startdefinition
      Un \defterme{simplexe} est une cellule dont les points extrémaux sont affinement indépendants.
    \stopdefinition

    \startremarque
      Les faces d'un simplexe de points extrémaux $a_0, \dots, a_k$ sont exactement les enveloppes convexes d'une partie des $a_i$. Ce sont en particulier des simplexes de dimension plus petite.
    \stopremarque


    \startdefinition
      Un \defterme{complexe simplicial} de $\reels^d$ est un ensemble localement fini $\Delta$ de simplexes tel que:
      \startlisteRom[compact]
        \startitem
          chaque face d'un simplexe de $\Delta$ est dans $\Delta$;
        \stopitem

        \startitem
          l'intersection de deux simplexes de $\Delta$ est une face commune aux deux.
        \stopitem
      \stoplisteRom
      Un \defterme{sous-complexe} de $\Delta$ est un complexe simplicial $\Sigma \subseteq \Delta$.
    \stopdefinition


    \startdiscussion[Squelettes]
      Étant donné un complexe simplicial $\Delta$ et un entier $n$, le \defterme{$n$-squelette} de $\Delta$ est le sous-complexe $\squelette{\Delta}{n}$ de $\Delta$ défini par:
      \startformula
        \squelette{\Delta}{n} = \comprehension{\sigma \in \Delta}{\dim{\sigma} \leq n}
      \stopformula
    \stopdiscussion


    \startdiscussion[Subdivision d'un complexe]
      La \defterme{réalisation géométrique} d'un complexe simplicial $\Delta$ est l'espace topologique:
      \startformula
        \realisation{\Delta} = \bigcup \Delta
      \stopformula
      On dit d'une partie d'un espace affine euclidien que c'est un \defterme{polyèdre} si elle est la réalisation d'un certain complexe. Une subdivision de $\Delta$ est un complexe simplicial $\Sigma$ tel que:
      \startformula
        \realisation{\Sigma} = \realisation{\Delta}
      \stopformula
      et dont chaque simplexe est inclus dans un simplexe de $\Delta$.
    \stopdiscussion


    \startdefinition
      Une \defterme{triangulation} d'un espace topologique $X$ est la donnée d'un complexe simplicial $\Delta$ et d'un homéomorphisme $X \iso \realisation{\Delta}$.
    \stopdefinition


    \starttheoreme
      Deux complexes simpliciaux qui réalisent le même polyèdre possèdent une subdivision commune.
    \stoptheoreme

    \startdem
      Soient $\Delta_1$ et $\Delta_2$ deux complexes simpliciaux de $\reels^d$ tels que:
      \startformula
        \realisation{\Delta_1} = \realisation{\Delta_2}
      \stopformula
      On va construire une suite croissante $\famille{\Sigma_k}{k \leq d}$ de complexes vérifiant pour tout $k$:
      \startlisteRom
        \startitem
          $\dim{\Sigma_k} \leq k$
        \stopitem

        \startitem
          si $\sigma_1 \in \Delta_1$ et $\sigma_2 \in \Delta_2$ sont d'intersection $k$-dimensionnelle:
          \startformula
            \sigma_1 \cap \sigma_2 = 
              \bigcup_{
                \startsubstack
                  \sigma \in \Sigma_k \NR
                  \sigma \subseteq \sigma_1 \cap \sigma_2 \NR
                \stopsubstack
              } \sigma
          \stopformula
        \stopitem

        \startitem
          $\realisation{\Sigma_k} = \displaystyle \bigcup_{\dim\p{\sigma_1 \cap \sigma_2} \leq k} \p{\sigma_1 \cap \sigma_2}$
        \stopitem
      \stoplisteRom

      On pose naturellement:
      \startformula
        \Sigma_0 = \comprehension{\sigma_1 \cap \sigma_2}{\dim\p{\sigma_1 \cap \sigma_2} = 0}
      \stopformula
      L'ensemble $\Delta_1 \cup \Delta_2$ étant localement fini, c'est également le cas de $\Sigma_0$. Étant constitué de simplexes réduits à des points, il s'agit d'un complexe simplicial.

      Supposons maintenant avoir construit le complexe $\Sigma_{k-1}$.\\
      Étant donnés des simplexes $\sigma_1 \in \Delta_1$ et $\sigma_2 \in \Delta_2$ tels que $\dim\p{\sigma_1 \cap \sigma_2} = k$, on note $b$ l'isobarycentre de la cellule $\sigma_1 \cap \sigma_2$ et:
      \startformula
        F\p{\sigma_1, \sigma_2} = \comprehension{\sigma \in \Sigma_{k-1}}{\sigma \subseteq \sigma_1 \cap \sigma_2}
      \stopformula
      L'ensemble $\Sigma_{k-1}$ étant localement fini et $\sigma_1 \cap \sigma_2$ compact, il vient que $F\p{\sigma_1, \sigma_2}$ est fini.


      \startassertion
        Chaque simplexe de $F\p{\sigma_1, \sigma_2}$ est contenu dans une face propre de $\sigma_1 \cap \sigma_2$.
      \stopassertion

      \startdem
        Soit $\sigma \in F\p{\sigma_1, \sigma_2}$. D'après le \cref[cor:faces_intersection_polytopes], on a:
        \startformula
          \p{\sigma_1 \cap \sigma_2} \cap \realisation{\Sigma_{k-1}} \subseteq \bigcup \facesExp\p{\sigma_1 \cap \sigma_2}
        \stopformula
        d'où $\sigma \subseteq \bigcup \facesExp\p{\sigma_1 \cap \sigma_2}$. Si $\lambda$ désigne la mesure de Lebesgue sur $\aff{\sigma}$, alors:
        \startformula
          0 < \lambda\p{\sigma} \leq
            \sum_{\omega} \lambda\p{\omega \cap \aff{\sigma}}
        \stopformula
        et il existe une face propre $\omega$ de $\sigma_1 \cap \sigma_2$ telle que $\lambda\p{\omega \cap \aff{\sigma}} > 0$. Par égalité des dimensions:
        \startformula \startalign
          \NC \aff{\sigma} \NC= \aff\p{\omega \cap \aff{\sigma}} \NR
          \NC \NC\subseteq \aff{\omega} \cap \aff{\sigma} \NR
          \NC \NC\subseteq \aff{\omega} \NR
        \stopalign \stopformula
        Finalement $\sigma \subseteq \aff{\omega} \cap \p{\sigma_1 \cap \sigma_2} = \omega$.
      \stopdem


      \startassertion
        Si $\sigma = \cellule{a_0, \dots, a_r}$ est un simplexe de $F\p{\sigma_1, \sigma_2}$, alors $\tau = \cellule{b, a_0, \dots, a_r}$ est un simplexe inclus dans $\sigma_1 \cap \sigma_2$ et $\tau \cap \realisation{\Sigma_{k-1}} = \sigma$.
      \stopassertion

      \startdem
        Étant donné $H$ un hyperplan d'appui de $\sigma_1 \cap \sigma_2$ dans $\aff\p{\sigma_1 \cap \sigma_2}$, on écrit:
        \startformula
          H = \comprehension{x}{\prodScal{x}{u} = \theta}
          \quad\text{et}\quad
          \pourTout{x \in \p{\sigma_1 \cap \sigma_2}} \prodScal{x}{u} \leq \theta
        \stopformula
        avec $u$ un vecteur non nul et $\theta$ un réel. Un argument de dimension montre l'existence d'un sommet de $\sigma_1 \cap \sigma_2$ hors de $H$, d'où $\prodScal{b}{u} < \theta$ et $b \nin H$. Mais $\sigma \subseteq \sigma_1 \cap \sigma_2$, et si $\tau$ désigne la cellule $\conv{\classe{b} \cup \sigma}$:
        \startformula
          \tau \cap H = \sigma \cap H
        \stopformula
        En particulier, si $B = \bigcup \facesExp\p{\sigma_1 \cap \sigma_2}$ alors :
        \startformula
          \tau \cap B = \sigma \cap B
        \stopformula
        Puis, $\tau$ étant inclus dans $\sigma_1 \cap \sigma_2$:
        \startformula\startalign
          \NC \tau \cap \realisation{\Sigma_{k-1}} \NC = \sigma \cap \realisation{\Sigma_{k-1}} \NR
          \NC \NC = \sigma
        \stopalign\stopformula
        % TODO : mettre référence
        Il ne reste plus qu'à vérifier que $\tau$ est un simplexe. L'(ASSERTION 1) montre qu'il existe une face propre $\omega$ de $\sigma_1 \cap \sigma_2$ telle que $\sigma \subseteq \omega$, or $b \nin \aff{\omega}$ d'après ce qui précède. Les points $b, a_0, \dots, a_r$ sont donc affinement indépendants.
      \stopdem


      En particulier, l'ensemble:
      \startformula
        \Sigma\p{\sigma_1, \sigma_2} = \comprehension{\conv \classe{b} \cup \sigma}{\sigma \in F\p{\sigma_1, \sigma_2}}
      \stopformula
      est formé de simplexes contenus dans $\sigma_1 \cap \sigma_2$ de dimension inférieure à $k$.


      \startassertion
        On a:
        \startformula
          \sigma_1 \cap \sigma_2 = \bigcup \Sigma\p{\sigma_1, \sigma_2}
        \stopformula
      \stopassertion

      \startdem
        Soit $x \in \p{\sigma_1 \cap \sigma_2} - \classe{b}$.\\
        Il existe, par compacité de $\sigma_1 \cap \sigma_2$, un réel $\lambda \geq 1$ maximal tel que $y = \lambda x + \p{1-\lambda} b \in \p{\sigma_1 \cap \sigma_2}$.\\
        Alors $y \in \frontiere\p{\sigma_1 \cap \sigma_2}$, donc $y$ appartient à une face propre $\tau_1 \cap \tau_2$ de $\sigma_1 \cap \sigma_2$. Par hypothèse de récurrence, $y$ appartient à un simplexe $\tau$ de $\Sigma_{k-1}$ contenu dans $\tau_1 \cap \tau_2$. Finalement $\conv{\classe{b} \cup \tau} \in \Sigma\p{\sigma_1, \sigma_2}$ contient $x$.
      \stopdem


      \startassertion
        Si $\sigma$ est face d'un simplexe de $\Sigma\p{\sigma_1, \sigma_2}$, alors:
        \startformula
          \sigma \in \Sigma_{k-1} \cup \Sigma\p{\sigma_1, \sigma_2}
        \stopformula
      \stopassertion

      \startdem
        Soit $\sigma$ face d'un simplexe $\tau$ de $\Sigma\p{\sigma_1, \sigma_2}$. L'ensemble des sommets de $\sigma$ est une partie de l'ensemble $\classe{b, a_0, \dots, a_r}$ des sommets de $\tau$.\\
        Mais $\cellule{a_0, \dots, a_r} \in \Sigma_{k-1}$, donc il suffit pour conclure de distinguer les cas selon si $b$ est ou non sommet de $\sigma$.
      \stopdem


      On définit alors:
      \startformula
        \Sigma_k = \Sigma_{k-1} \cup \bigcup_{\dim\p{\sigma_1 \cap \sigma_2} = k} \Sigma\p{\sigma_1, \sigma_2}
      \stopformula
      Soit $U$ un ouvert qui intersecte un nombre fini de simplexes de $\Delta_1 \cup \Delta_2$. Le nombre de couples $\p{\sigma_1, \sigma_2}$ tels que $U$ rencontre au moins un simplexe de $\Sigma\p{\sigma_1, \sigma_2}$ est fini, donc $U$ intersecte un nombre fini de simplexes de $\Sigma_k - \Sigma_{k-1}$. Finalement, l'ensemble $\Sigma_k = \Sigma_{k-1} \cup \p{\Sigma_k - \Sigma_{k-1}}$ est localement fini.


      \startassertion
        Si $\sigma \in \Sigma\p{\sigma_1, \sigma_2}$ et $\tau \in \Sigma\p{\tau_1, \tau_2}$, alors $\sigma \cap \tau \in \Sigma_k$.
      \stopassertion

      % TODO : rajouter les références
      \startdem
        D'après (ASSERTION 2), les ensembles:
        \startformula
          \sigma' = \sigma \cap \realisation{\Sigma_{k-1}}
          \quad\text{et}\quad
          \tau' = \tau \cap \realisation{\Sigma_{k-1}}
        \stopformula
        sont des simplexes de $\Sigma_{k-1}$. On distingue alors deux cas:
        \startlisteEnum
          \startitem
            si $\sigma_1 \cap \sigma_2$ et $\tau_1 \cap \tau_2$ sont une même cellule d'isobarycentre $b$, alors:
            \startformula
              \sigma = \conv{\classe{b} \cup \sigma'}
              \quad\text{et}\quad
              \tau = \conv{\classe{b} \cup \tau'}
            \stopformula
            Soit $a \in \p{\sigma \cap \tau} - \classe{b}$. Quitte à échanger $\sigma$ et $\tau$, on peut écrire:
            \startformula\startalign
              \NC a \NC = \lambda b + \p{1-\lambda} y \NR
              \NC \NC = \mu b + \p{1-\mu} z \NR
            \stopalign\stopformula
            avec $0 \leq \lambda \leq \mu < 1$, $y \in \sigma'$ et $z \in \tau'$, et alors:
            \startformula
              y = \frac{\mu - \lambda}{1-\lambda} b + \frac{1-\mu}{1-\lambda} z \in \tau
            \stopformula
            En particulier $y \in \sigma' \cap \tau = \sigma' \cap \tau'$ et $a \in \conv{\classe{b} \cup \p{\sigma' \cap \tau'}}$.\\
            On en déduit $\sigma \cap \tau \subseteq \conv{\classe{b} \cup \p{\sigma' \cap \tau'}}$, puis:
            \startformula
              \sigma \cap \tau = \conv{\classe{b} \cup \p{\sigma' \cap \tau'}}
            \stopformula
            et $\sigma \cap \tau \in \Sigma\p{\sigma_1, \sigma_2}$.
          \stopitem

          \startitem
            sinon $\p{\sigma_1 \cap \tau_1} \cap \p{\sigma_2 \cap \tau_2}$ est une face propre des cellules $\sigma_1 \cap \sigma_2$ et $\tau_1 \cap \tau_2$, donc est de dimension inférieure à $k-1$. En particulier:
            \startformula
              \sigma \cap \tau \subseteq \p{\sigma_1 \cap \tau_1} \cap \p{\sigma_2 \cap \tau_2} \subseteq \realisation{\Sigma_{k-1}}
            \stopformula
            puis:
            \startformula\startalign
              \NC \sigma \cap \tau \NC = \p{\sigma \cap \realisation{\Sigma_{k-1}}} \cap \p{\tau \cap \realisation{\Sigma_{k-1}}} \NR
              \NC \NC = \sigma' \cap \tau' \NR
            \stopalign\stopformula
            On en déduit $\sigma \cap \tau \in \Sigma_{k-1}$.
          \stopitem
        \stoplisteEnum
      \stopdem


      % TODO : mettre références
      Les (ASSERTIONS 4 & 5) montrent finalement que l'ensemble localement fini $\Sigma_k$ est un complexe simplicial, et:
      \startformula
        \realisation{\Sigma_k} = \bigcup_{\dim\p{\sigma_1 \cap \sigma_2} \leq k} \p{\sigma_1 \cap \sigma_2}
      \stopformula
      en vertu de (ASSERTION 3).
    \stopdem



    \startdefinition
      Une suite $\famille{\Delta_i}{i}$ de subdivisions d'un complexe simplicial $\Delta$ est dite:
      \startlisteEnum[fin]
        \startitem
          \defterme{fine} si le supremum des diamètres des simplexes de $\Delta_i$ tend vers $0$ lorsque $i \to \infty$;
        \stopitem

        \startitem
          \defterme{cristalline} s'il existe une constante $\delta > 0$ telle que $\corp\p{\sigma} \geq \delta$ pour chaque simplexe $\sigma$ de $\Delta_i$ pour tout $i$.
        \stopitem
      \stoplisteEnum
      
    \stopdefinition


    \startdiscussion[Triangulations cristallines de $\reels^n$]
      Étant donnée une permutation $f \in \grpSym_n$, on définit:
      \startformula
        \sigma_f = \comprehension{\p{x_1, \dots, x_n} \in \segment{0}{1}^n}{x_{f\p{1}} \geq \dots \geq x_{f\p{n}}}
      \stopformula
      C'est un simplexe, car:
      \startformula
        \sigma_f = \conv\famille[\big]{\sum_{i \geq k} e_{f\p{i}}}{0 \leq k \leq n}
      \stopformula
      où $\p{e_1, \dots, e_n}$ désigne la base canonique de $\reels^n$.
      % TODO : à finir
    \stopdiscussion


    \startcorollaire[suites_subdivs_complexeSimplicial]
      Tout complexe simplicial fini possède des suites de subdivisions fines et cristallines.
    \stopcorollaire

    \startdem
      % TODO: dém. du lemme de subdivision (utilise la triangulation cristalline)
    \stopdem


    \startdefinition
      Si $X$ est un polyèdre, alors une application $\application{f}{X \fleche \reels^n}$ est dite \defterme{linéaire par morceaux} (resp. \defterme{lisse par morceaux}) s'il existe une triangulation $\Delta$ de $X$ telle que la restriction de $f$ à chaque simplexe de $\Delta$ est affine (resp. lisse).
      
      Un \defterme{plongement lisse par morceaux} est une injection $X \fleche \reels^n$ lisse par morceaux dont la restriction à chaque simplexe de la triangulation est une immersion lisse.
    \stopdefinition

    \startdefinition
      Étant donné un complexe simplicial $\Delta$ et une application $\application{f}{\realisation{\Delta} \fleche \reels^n}$, la \defterme{sécante} de $f$ relativement à $\Delta$ est l'unique application $\application{\secante{f}{\Delta}}{\realisation{\Delta} \fleche \reels^n}$ linéaire par morceaux telle que:
      \startlisteRom[fin]
        \startitem
          $f$ et $\secante{f}{\Delta}$ coïncident sur les sommets de $\Delta$;
        \stopitem

        \startitem
          la restriction de $\secante{f}{\Delta}$ à simplexe de $\Delta$ est affine.
        \stopitem
      \stoplisteRom
    \stopdefinition


    \startproposition[approximation_C1][Approximation $\creg{1}$]
      Si $X$ est un polyèdre compact et $f$ une application $X \fleche \reels$ lisse par morceaux, alors il existe une suite $\famille{f_i}{i \in \naturels}$ d'applications linéaires par morceaux $X \fleche \reels$ qui converge vers $f$ au sens de la topologie $\creg{1}$.
    \stopproposition

    \startdem
      Soit $\Delta$ une triangulation de $X$. C'est un complexe simplicial fini par compacité, et il existe d'après le \cref[cor:suites_subdivs_complexeSimplicial] une suite $\famille{\Delta_i}{i \in \naturels}$ de subdivisions de $\Delta$ à la fois fine et cristalline. On note dans la suite $f_i = \secante{f}{\Delta_i}$ pour tout $i \in \naturels$.

      La suite $\famille{\Delta_i}{}$ est fine, donc la continuité uniforme de $f$ sur $X$ assure la convergence $\creg{0}$-uniforme de la suite $\famille{f_i}{}$ vers $f$. Il s'agit maintenant de montrer la convergence uniforme des $f_i'$ vers $f'$.

      On fixe un $\epsilon > 0$. Pour $i$ suffisamment grand, on a pour tout simplexe $\sigma$ de $\Delta_i$:
      \startformula
        \pourTout{x,y \in \sigma} \norme{\restriction{f'}{\sigma}\p{x} - \restriction{f'}{\sigma}\p{y}} \leq \epsilon
      \stopformula
      Soit $i$ un tel indice et $\sigma = \cellule{a_0, \dots, a_s} \in \Delta_i$.\\
      Si $x \in \sigma$, alors $\restriction{f_i'}{\sigma}\p{x} = \restriction{f_i'}{\sigma}\p{a_0}$ du fait du caractère affine de $f_i$ sur $\sigma$ et :
      \startformula\startalign
        \NC \norme{\restriction{f_i'}{\sigma}\p{x} - \restriction{f'}{\sigma}\p{x}} \NC \leq \norme{\restriction{f_i'}{\sigma}\p{a_0} - \restriction{f'}{\sigma}\p{a_0}} + \norme{\restriction{f'}{\sigma}\p{a_0} - \restriction{f'}{\sigma}\p{x}} \NR
        \NC \NC \leq \norme{A\p{a_0}} + \epsilon \NR
      \stopalign\stopformula
      où $A = \restriction{f_i'}{\sigma}\p{a_0} - \restriction{f'}{\sigma}\p{a_0}$.

      Pour $1 \leq i \leq s$, on a:
      \startformula\startalign
        \NC A\p{a_i - a_0} \NC = f\p{a_i} - f\p{a_0} - \restriction{f'}{\sigma}\p{a_0} \cdot \p{a_i - a_0} \NR
        \NC \NC = \p{\restriction{f'}{\sigma}\p{c_i} -  \restriction{f'}{\sigma}\p{a_0}} \cdot \p{a_i - a_0} \NR
      \stopalign\stopformula
      où $c_i$ est un point du segment $\segment{a_0}{a_i} \subseteq \sigma$ dont l'existence est garantie par le théorème des accroissements finis. On en déduit $\abs{A\p{a_i - a_0}} \leq d\p{\sigma} \epsilon$, puis:
      \startformula
        \pourTout{y \in \sigma} \abs{A\p{y-a_0}} \leq d\p{\sigma} \epsilon
      \stopformula
      par linéarité.

      Mais alors, si $b$ désigne l'isobarycentre de $\sigma$ et si $u \in \aff{\sigma}$ est de norme $r\p{\sigma}$, les points $b$ et $b+u$ appartiennent à $\sigma$ et:
      \startformula
        \abs{A\p{u}} \leq \abs{A\p{b+u-a_0}} + \abs{A\p{b-a_0}} \leq 2 d\p{\sigma} \epsilon
      \stopformula
      d'où $\norme{A} \leq 2 \epsilon / \corp\p{\sigma}$. Le caractère cristallin de la suite $\famille{\Delta_i}{}$ permet de conclure.
    \stopdem


  \stopsection


\stopchapter

\stopcomponent
