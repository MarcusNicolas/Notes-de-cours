\startcomponent c_GéométrieConvexe
\product prd_GéométrieRDP

  \startchapter[title={Un peu de géométrie convexe}]


    \startsection[title={Ensembles convexes}]

      \startdiscussion
        On rappelle, étant donné un espace affine $A$ sur $\reels$, qu'une partie $C \subseteq A$ est dite \defterme{convexe} lorsque le segment $\segment{x}{y}$ est contenu dans $C$ pour tout couple $\p{x, y}$ de points de $C$.
      \stopdiscussion


      \starttheoreme[Carathéodory]
        Tout point de l'enveloppe convexe d'une partie $A$ de $\reels^d$ est dans l'enveloppe convexe de $d+1$ points de $A$.
      \stoptheoreme


      \startdem
        Étant donné $y \in \enveloppeConvexe{A}$, il existe un entier $r$ minimal, des réels positifs $\lambda_i$ et des points $x_i$ de $A$ vérifiant :
        \startformula
          y = \sum_{i=0}^r \lambda_i x_i
          \quad\text{et}\quad
          \sum_{i=0}^r \lambda_i = 1
        \stopformula
        Si $r \geq d+1$, alors la famille $x_0, \dots, x_r$ n'est pas affinement indépendante. Autrement dit, il existe des réels $\mu_0, \dots, \mu_r$ non tous nuls tels que :
        \startformula
          \sum_{i=0}^r \mu_i x_i = 0
          \quad\text{et}\quad
          \sum_{i=0}^r \mu_i = 0
        \stopformula
        On peut supposer avoir $\mu_r > 0$ et $\lambda_r / \mu_r < \lambda_i / \mu_i$ pour tout $i$ tel que $\mu_i > 0$. Alors :
        \startformula
          x_j = -\sum_{i=0}^{r-1} \frac{\mu_i}{\mu_r} x_i
          \quad\text{puis}\quad
          y = \sum_{i=0}^{r-1}
            \p{\lambda_i - \frac{\lambda_r}{\mu_r} \mu_i} x_i
        \stopformula
        et on vérifie immédiatement que :
        \startformula
          \sum_{i=0}^{r-1} \p{\lambda_i - \frac{\lambda_r}{\mu_r} \mu_i} = 1
          \qquad\text{et}\qquad
          \pourTout{i < r} \lambda_i - \frac{\lambda_r}{\mu_r} \mu_i \geq 0
        \stopformula
        C'est absurde car $r$ est supposé minimal, donc $r \leq d$.
      \stopdem


      \startdefinition
        Soit $K$ une partie convexe fermée de $\reels^d$ de dimension $d$. Une \defterme{face exposée} de $K$ est une partie de $K$ vide ou de la forme $H \cap K$ avec $H$ hyperplan d'appui de $K$.
        \blank[halfline]
        L'ensemble des faces exposées de $K$ est noté $\facesExp\p{K}$.
      \stopdefinition


      \startproposition
        Soit $K$ un convexe fermé de dimension $d$.
        \startlisteOrd[style=compact][fin]
          \startitem
            Les éléments de $\facesExp\p{K}$ sont des parties convexes fermées de $\frontiere{K}$.
          \stopitem
          \startitem
            L'ensemble $\facesExp\p{K}$ est stable par intersections finies.
          \stopitem
        \stoplisteOrd
      \stopproposition

      \startdem
        \startlisteOrd
          \startitem
            Clair.
          \stopitem

          \startitem
            Soient $F_1, \dots, F_m \in \facesExp\p{K}$. On note :
            \startformula
              F = \bigcap_{i=1}^m F_i
            \stopformula
            Si $F$ est vide alors il n'y a rien à faire. Sinon, quitte à translater $K$, on peut supposer que $0 \in F$. Il existe alors pour tout $i$ un vecteur non nul $u_i$ tel que, si $H_i$ désigne l'hyperplan d'équation $\produitScalaire{x}{u_i} = 0$ :
            \startformula
              F_i = H_i \cap K
              \quad\text{et}\quad
              \pourTout{x \in K} \produitScalaire{x}{u_i} \leq 0
            \stopformula
            Alors, si $u = \sum u_i$ (MONTRER QUE $u \neq 0$) et si $H$ désigne l'hyperplan d'équation $\produitScalaire{x}{u} \leq 0$ :
            \startformula
              F = H \cap K
              \quad\text{et}\quad
              \pourTout{x \in K} \produitScalaire{x}{u} \leq 0
            \stopformula
            Finalement, $H$ est un hyperplan d'appui de $K$ en $0$ et $F \in \facesExp\p{K}$.
          \stopitem
        \stoplisteOrd
      \stopdem

    \stopsection



    \startsection[title={Polytopes}]
      
    \stopsection


    \startsection[title={Cellulations}]

    \stopsection



    \startsection[title={Simplexes et complexes simpliciaux}]
      \startdefinition
        Un \defterme{simplexe} est une cellule dont les points extrémaux sont affinement indépendants.
      \stopdefinition

      METTRE LA REMARQUE

      \startdefinition
        Un \defterme{complexe simplicial} de $\reels^d$ est un ensemble localement fini $\Delta$ de simplexes tel que :
        \startlisteOrd[style=compact][r]
          \startitem
            chaque face d'un simplexe de $\Delta$ est dans $\Delta$;
          \stopitem

          \startitem
            l'intersection de deux simplexes de $\Delta$ est une face commune aux deux.
          \stopitem
        \stoplisteOrd
        Un \defterme{sous-complexe} de $\Delta$ est un complexe simplicial $\Sigma \subseteq \Delta$.
      \stopdefinition

      METTRE L'EXEMPLE

      \startdefinition
        Si $\Delta$ est un complexe simplicial, on définit la réalisation géométrique de $\Delta$ comme étant l'espace topologique :
        \startformula
          \realisation{\Delta} = \bigcup \Delta
        \stopformula
        Une \defterme{triangulation} d'un espace topologique $X$ est alors la donnée d'un complexe simplicial $\Delta$ et d'un homéomorphisme $X \iso \realisation{\Delta}$.
      \stopdefinition

      \startdefinition
        Soit $\Delta$ un complexe simplicial.\\
        Une \defterme{subdivision} de $\Delta$ est un complexe simplicial $\Sigma$ tel que :
        \startformula
          \realisation{\Sigma} = \realisation{\Delta}
        \stopformula
        et dont chaque simplexe est inclus dans un simplexe de $\Delta$.
      \stopdefinition


      \starttheoreme
        Deux complexes simpliciaux qui réalisent le même polyèdre (A DEFINIR) possèdent une subdivision commune.
      \stoptheoreme

      \startdem
        Soient $\Delta_1$ et $\Delta_2$ deux complexes simpliciaux de $\reels^d$ tels que :
        \startformula
          \realisation{\Delta_1} = \realisation{\Delta_2}
        \stopformula
        On va construire une suite croissante $\famille{\Sigma_k}{k \leq d}$ de complexes vérifiant pour tout $k$ :
        \startlisteOrd[style=compact][r]
          \startitem
            $\dim{\Sigma_k} \leq k$
          \stopitem

          \startitem
            si $\sigma_1 \in \Delta_1$ et $\sigma_2 \in \Delta_2$ sont d'intersection $k$-dimensionnelle :
            \startformula
              \sigma_1 \cap \sigma_2 = 
                \bigcup_{
                  \startsubstack
                    \sigma \in \Sigma_k \NR
                    \sigma \subseteq \sigma_1 \cap \sigma_2 \NR
                  \stopsubstack
                } {\sigma}
            \stopformula
          \stopitem

          \startitem
            $\realisation{\Sigma_k} = \displaystyle \bigcup_{\dim\p{\sigma_1 \cap \sigma_2} \leq k} \p{\sigma_1 \cap \sigma_2}$
          \stopitem
        \stoplisteOrd

        On pose naturellement :
        \startformula
          \Sigma_0 = \comprehension{\frac{x}{y}}{dz}
        \stopformula
      \stopdem


      \startdiscussion[Subdivision cristalline]
        Oui
      \stopdiscussion
    \stopsection

  \stopchapter

\stopcomponent
