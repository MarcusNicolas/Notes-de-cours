\startcomponent c_GéométrieConvexe
\product prd_GéométrieRDP

\startchapter[title={Un peu de géométrie convexe}]

  %%%
  \startsection[title={Faces d'un ensemble convexe}]

    \startdiscussion[Convexité]
      On rappelle, étant donné un espace affine $A$ sur $\reels$, qu'une partie $C \subseteq A$ est dite \defterme{convexe} lorsque le segment $\segment{x}{y}$ est contenu dans $C$ pour tout couple $\p{x, y}$ de points de $C$. La \defterme{dimension} de $C$, notée $\dim{C}$, est alors la dimension de son enveloppe affine $\aff{C}$. L'intérieur et la frontière de $C$ sont toujours considérés relativement à l'espace topologique $\aff{C}$.
    \stopdiscussion


    \startremarque
      Un convexe non vide est d'intérieur non vide.
    \stopremarque


    \starttheoreme[Carathéodory]
      Tout point de l'enveloppe convexe d'une partie $A$ d'un espace affine de dimension $d$ est barycentre de $d+1$ points de $A$.
    \stoptheoreme


    \startdem
      Étant donné $y \in \conv{A}$, il existe un entier $r$ minimal, des réels positifs $\lambda_i$ et des points $x_i$ de $A$ vérifiant :
      \startformula
        y = \sum_{i=0}^r \lambda_i x_i
        \quad\text{et}\quad
        \sum_{i=0}^r \lambda_i = 1
      \stopformula
      Si $r \geq d+1$, alors la famille $x_0, \dots, x_r$ n'est pas affinement indépendante. Autrement dit, il existe des réels $\mu_0, \dots, \mu_r$ non tous nuls tels que :
      \startformula
        \sum_{i=0}^r \mu_i x_i = 0
        \quad\text{et}\quad
        \sum_{i=0}^r \mu_i = 0
      \stopformula
      On peut supposer avoir $\mu_r > 0$ et $\lambda_r / \mu_r < \lambda_i / \mu_i$ pour tout $i$ tel que $\mu_i > 0$. Alors :
      \startformula
        x_j = -\sum_{i=0}^{r-1} \frac{\mu_i}{\mu_r} x_i
        \quad\text{puis}\quad
        y = \sum_{i=0}^{r-1}
          \p{\lambda_i - \frac{\lambda_r}{\mu_r} \mu_i} x_i
      \stopformula
      et on vérifie immédiatement que :
      \startformula
        \sum_{i=0}^{r-1} \p{\lambda_i - \frac{\lambda_r}{\mu_r} \mu_i} = 1
        \qquad\text{et}\qquad
        \pourTout{i < r} \lambda_i - \frac{\lambda_r}{\mu_r} \mu_i \geq 0
      \stopformula
      C'est absurde car $r$ est supposé minimal, donc $r \leq d$.
    \stopdem


    On travaille implicitement dans le reste de chapitre dans un espace affine euclidien ambiant $E$.

    \startdefinition
      Soit $K$ un convexe fermé. Une \defterme{face exposée} de $K$ est une partie de $K$ vide ou de la forme $H \cap K$ avec $H$ hyperplan d'appui de $K$ dans $\aff{K}$.
      
      L'ensemble des faces exposées de $K$ est noté $\facesExp\p{K}$.\\
      Ses éléments maximaux pour la relation d'inclusion forment l'ensemble $\facesMax\p{K}$.
    \stopdefinition


    \startproposition
      Étant donné un convexe fermé $K$:
      \startlisteNum[fin]
        \startitem
          Les éléments de $\facesExp\p{K}$ sont des parties convexes fermées de $\frontiere{K}$.
        \stopitem
        \startitem
          L'ensemble $\facesExp\p{K}$ est stable par intersections finies.
        \stopitem
      \stoplisteNum
    \stopproposition

    \startdem
      \startlisteNum
        \startitem
          Clair.
        \stopitem

        \startitem
          Soient $F_1, \dots, F_m \in \facesExp\p{K}$. On note :
          \startformula
            F = \bigcap_{i=1}^m F_i
          \stopformula
          Si $F$ est vide alors il n'y a rien à faire. Sinon on fixe $a \in F$. Il existe alors pour tout $i$ un vecteur non nul $u_i$ tel que, si $H_i$ désigne l'hyperplan d'équation $\prodScal{x-a}{u_i} = 0$ :
          \startformula
            F_i = H_i \cap K
            \quad\text{et}\quad
            \pourTout{x \in K} \prodScal{x-a}{u_i} \leq 0
          \stopformula
          Le vecteur $u = \sum_i u_i$ est non nul, car si $b \in \interieur{K}$ alors $\prodScal{b-a}{u} < 0$.\\
          Finalement, si $H$ désigne l'hyperplan d'équation $\prodScal{x}{u} = 0$:
          \startformula
            F = H \cap K
            \quad\text{et}\quad
            \pourTout{x \in K} \prodScal{x}{u} \leq 0
          \stopformula
          et $H$ est un hyperplan d'appui de $K$ en $a$, d'où $F \in \facesExp\p{K}$.
        \stopitem
      \stoplisteNum
    \stopdem


    On fixe dans la suite un convexe fermé $K$.

    \startlemme
      Si $F \subseteq F'$ sont deux faces exposées de $K$ de même dimension, alors $F = F'$.
    \stoplemme

    \startdem
      % TODO
    \stopdem


    \startproposition
      Si $K$ n'est pas réduit à un point, alors :
      \startformula
        \frontiere{K} = \bigcup \facesMax\p{K}
      \stopformula
    \stopproposition

    \startdem
      % TODO
    \stopdem


    \startremarque
      ON VOUDRAIT LA PROP DE TRANSITIVITÉ POUR LES FACES EXPOSEES MAIS C'EST FAUX (mettre un diagramme) % TODO
    \stopremarque


    \startdefinition
      Une \defterme{face} de $K$ est une partie $F$ de $K$ telle qu'il existe une suite de convexes fermés :
      \startformula
        K = F_0 \supseteq \dots \supseteq F_k = F
      \stopformula
      telle que $F_i \in \facesExp\p{F_{i-1}}$ pour $1 \leq i \leq k$.

      On désigne par $\faces\p{K}$ l'ensemble des faces de $K$.
    \stopdefinition


    \startremarque
      L'ensemble des faces propres maximales de $K$ est exactement $\facesMax\p{K}$.
    \stopremarque

    \startdiscussion[Transitivité]
      Si $F$ est une face de $K$
    \stopdiscussion


    \startcorollaire
      Si $F$ est une face de $K$, alors $\faces\p{F} = \faces\p{K} \cap \parties\p{F}$.
    \stopcorollaire


    \startproposition
      Si $F$ est une face de $K$, alors $\ptExt{F} = F \cap \ptExt{K}$.
    \stopproposition

    \startdem
      % TODO
    \stopdem


    \startcorollaire
      Les points extrémaux de $K$ sont exactement les faces singletons.
    \stopcorollaire

    \startdem
      % TODO
    \stopdem



  \stopsection


  %%%
  \startsection[title={Polytopes convexes}]

    \startdefinition
      Un \defterme{polytope convexe} est un convexe non vide qui est intersection finie de demi-espaces fermés dans son enveloppe affine.\\
      Les $0$-faces et $1$-faces d'un polytope sont respectivement appelés \defterme{sommets} et \defterme{arêtes}.
    \stopdefinition


    \startdiscussion
      Étant donné un polytope convexe $K$, on fixe une famille minimale de demi-espaces fermés $D_1, \dots, D_r$ de $\aff{K}$ vérifiant :
      \startformula
        K = \bigcap_{i=1}^r D_i
      \stopformula
      et on écrit :
      \startformula
        D_i = \comprehension{x}{\prodScal{x-a}{u_i} \leq \theta_i}
        \quad\text{et}\quad
        F_i = \frontiere{D_i} \cap K
      \stopformula
      pour tout $1 \leq i \leq r$.


      \startsousdiscussion
        Finalement :
        \startformula
          \frontiere{K} = \bigcup_{i=1}^r F_i
        \stopformula
      \stopsousdiscussion


      \startsousdiscussion
        ALORS $\facesMax\p{K} = \classe{F_1, \dots, F_r}$.
      \stopsousdiscussion


      \startsousdiscussion
        Finalement, étant donné un indice $1 \leq i \leq r$, tout élément de $\facesMax\p{F_i}$ est intersection de deux éléments de $\facesMax\p{K}$.
      \stopsousdiscussion
    \stopdiscussion


    \startproposition
      Une face propre de $K$ est intersection finie d'éléments de $\facesMax\p{K}$.
    \stopproposition

    \startdem
      On raisonne par récurrence sur la dimension $d$ de $K$:
      \startlisteEnum
        \startitem
          si $K$ est réduit à un point, alors l'énoncé est vide.
        \stopitem

        \startitem
          % TODO
          si $d \geq 1$, ALORS
        \stopitem
      \stoplisteEnum
    \stopdem


    On déduit immédiatement de ce qui précède :

    \starttheoreme
      Si $K$ est un polytope convexe, alors les faces exposées de $K$ sont exactement ses faces propres et sont en nombre fini.
    \stoptheoreme



    \startdiscussion
      DIGRESSION, ORGANISER LES PROPS SUIVANTES
    \stopdiscussion


    \startproposition
      Soit $K$ un polytope convexe.\\
      Si $D_1, \dots, D_r$ sont des demi-espaces fermés de l'espace $E$ ambiant tels que :
      \startformula
        K = \bigcap_{i=1}^r D_i
      \stopformula
      En notant $F_i = \frontiere{D_i} \cap K$ pour tout $1 \leq i \leq r$, les faces de $K$ sont alors exactement les intersections dans $K$ d'un nombre fini de $F_i$.
    \stopproposition

    \startdem
      % TODO
    \stopdem


    \startcorollaire
      Si $K_1$ et $K_2$ sont deux polytopes convexes, alors :
      \startformula
        \faces\p{K_1 \cap K_2} = \comprehension{F_1 \cap F_2}{\p{F_1, F_2} \in \faces\p{K_1} \times \faces\p{K_2}}
      \stopformula
    \stopcorollaire

    \startdem
      On peut écrire dans $E$ :
      \startformula
        K_1 = \bigcap_{i=1}^r D_i
        \quad\text{et}\quad
        K_2 = \bigcap_{i=r+1}^{r+s} D_i
      \stopformula
      où les $D_i$ sont des demi-espaces fermés. Alors :
      \startformula
        K_1 \cap K_2 = \bigcap_{i=1}^{r+s} D_i
      \stopformula
      et la (PROP PRECEDENTE) permet de conclure.
    \stopdem


  \stopsection


  %%%
  \startsection[title={Cellules linéaires}]

    \startdefinition
      Une \defterme{cellule linéaire} est l'enveloppe convexe d'un nombre fini de points.
    \stopdefinition


    \startdiscussion
      DISCUTER UN PEU\\ % TODO
    \stopdiscussion


    \startlemme
      Si $K$ est une cellule, alors tout point d'une $k$-face de $K$ appartient à l'enveloppe convexe de $k+1$ points extrémaux de $K$.
    \stoplemme

    \startdem
      % TODO
    \stopdem


    Le résultat suivant est fondamental :

    \starttheoreme
      Les cellules sont exactement les polytopes bornés.
    \stoptheoreme

    \startdem
      % TODO
    \stopdem


    \startdefinition
      Un \defterme{complexe} est un ensemble localement fini $\Delta$ de cellules tel que :
      \startlisteRom[compact]
        \startitem
          si $K_1$ et $K_2$ sont deux cellules distinctes de $\Delta$ alors $\interieur{K_1}$ et $\interieur{K_2}$ sont disjoints;
        \stopitem

        \startitem
          si $K \in \Delta$, alors $\frontiere{K}$ est réunion de cellules de $\Delta$.
        \stopitem
      \stoplisteRom
      Un \defterme{sous-complexe} de $\Delta$ est un complexe $\Sigma \subseteq \Delta$.
    \stopdefinition


  \stopsection


  %%%
  \startsection[title={Simplexes et complexes simpliciaux}]
    \startdefinition
      Un \defterme{simplexe} est une cellule dont les points extrémaux sont affinement indépendants.
    \stopdefinition

    METTRE LA REMARQUE

    \startdefinition
      Un \defterme{complexe simplicial} de $\reels^d$ est un ensemble localement fini $\Delta$ de simplexes tel que :
      \startlisteRom[compact]
        \startitem
          chaque face d'un simplexe de $\Delta$ est dans $\Delta$;
        \stopitem

        \startitem
          l'intersection de deux simplexes de $\Delta$ est une face commune aux deux.
        \stopitem
      \stoplisteRom
      Un \defterme{sous-complexe} de $\Delta$ est un complexe simplicial $\Sigma \subseteq \Delta$.
    \stopdefinition

    METTRE L'EXEMPLE

    \startdefinition
      Si $\Delta$ est un complexe simplicial, on définit la réalisation géométrique de $\Delta$ comme étant l'espace topologique :
      \startformula
        \realisation{\Delta} = \bigcup \Delta
      \stopformula
      Une \defterme{triangulation} d'un espace topologique $X$ est alors la donnée d'un complexe simplicial $\Delta$ et d'un homéomorphisme $X \iso \realisation{\Delta}$.
    \stopdefinition

    \startdefinition
      Soit $\Delta$ un complexe simplicial.\\
      Une \defterme{subdivision} de $\Delta$ est un complexe simplicial $\Sigma$ tel que :
      \startformula
        \realisation{\Sigma} = \realisation{\Delta}
      \stopformula
      et dont chaque simplexe est inclus dans un simplexe de $\Delta$.
    \stopdefinition


    \starttheoreme
      Deux complexes simpliciaux qui réalisent le même polyèdre (A DEFINIR) possèdent une subdivision commune.
    \stoptheoreme

    \startdem
      Soient $\Delta_1$ et $\Delta_2$ deux complexes simpliciaux de $\reels^d$ tels que :
      \startformula
        \realisation{\Delta_1} = \realisation{\Delta_2}
      \stopformula
      On va construire une suite croissante $\famille{\Sigma_k}{k \leq d}$ de complexes vérifiant pour tout $k$ :
      \startlisteRom[compact]
        \startitem
          $\dim{\Sigma_k} \leq k$
        \stopitem

        \startitem
          si $\sigma_1 \in \Delta_1$ et $\sigma_2 \in \Delta_2$ sont d'intersection $k$-dimensionnelle :
          \startformula
            \sigma_1 \cap \sigma_2 = 
              \bigcup_{
                \startsubstack
                  \sigma \in \Sigma_k \NR
                  \sigma \subseteq \sigma_1 \cap \sigma_2 \NR
                \stopsubstack
              } \sigma
          \stopformula
        \stopitem

        \startitem
          $\realisation{\Sigma_k} = \displaystyle \bigcup_{\dim\p{\sigma_1 \cap \sigma_2} \leq k} \p{\sigma_1 \cap \sigma_2}$
        \stopitem
      \stoplisteRom

      On pose naturellement :
      \startformula
        \Sigma_0 = \comprehension{\sigma_1 \cap \sigma_2}{\dim\p{\sigma_1 \cap \sigma_2} = 0}
      \stopformula
      L'ensemble $\Delta_1 \cup \Delta_2$ étant localement fini, c'est également le cas de $\Sigma_0$. Étant constitué de simplexes réduits à des points, il s'agit d'un complexe simplicial.

      Supposons maintenant avoir construit le complexe $\Sigma_{k-1}$.\\
      Étant donnés des simplexes $\sigma_1 \in \Delta_1$ et $\sigma_2 \in \Delta_2$ tels que $\dim\p{\sigma_1 \cap \sigma_2} = k$, on note $b$ l'isobarycentre de la cellule $\sigma_1 \cap \sigma_2$ et :
      \startformula
        F\p{\sigma_1, \sigma_2} = \comprehension{\sigma \in \Sigma_{k-1}}{\sigma \subseteq \sigma_1 \cap \sigma_2}
      \stopformula
      On commence par remarquer, $\Sigma_{k-1}$ étant localement fini et $\sigma_1 \cap \sigma_2$ compact, que l'ensemble $F\p{\sigma_1, \sigma_2}$ est fini.


      \startassertion
        Chaque simplexe de $F\p{\sigma_1, \sigma_2}$ est contenu dans une face propre de $\sigma_1 \cap \sigma_2$.
      \stopassertion

      \startdem
        Soit $\sigma \in F\p{\sigma_1, \sigma_2}$. D'après le (COROLLAIRE 1.14), on a :
        \startformula
          \p{\sigma_1 \cap \sigma_2} \cap \realisation{\Sigma_{k-1}} \subseteq \bigcup \facesExp\p{\sigma_1 \cap \sigma_2}
        \stopformula
        d'où $\sigma \subseteq \bigcup \facesExp\p{\sigma_1 \cap \sigma_2}$. Si $\lambda$ désigne la mesure de Lebesgue sur $\aff{\sigma}$, alors :
        \startformula
          0 < \lambda\p{\sigma} \leq
            \sum_{\omega} \lambda\p{\omega \cap \aff{\sigma}}
        \stopformula
        et il existe une face propre $\omega$ de $\sigma_1 \cap \sigma_2$ telle que $\lambda\p{\omega \cap \aff{\sigma}} > 0$. Par égalité des dimensions :
        \startformula \startalign
          \NC \aff{\sigma} \NC= \aff\p{\omega \cap \aff{\sigma}} \NR
          \NC \NC\subseteq \aff{\omega} \cap \aff{\sigma} \NR
          \NC \NC\subseteq \aff{\omega} \NR
        \stopalign \stopformula
        Finalement $\sigma \subseteq \aff{\omega} \cap \p{\sigma_1 \cap \sigma_2} = \omega$.
      \stopdem


      \startassertion
        Si $\sigma = \cellule{a_0, \dots, a_r}$ est un simplexe de $F\p{\sigma_1, \sigma_2}$, alors $\tau = \cellule{b, a_0, \dots, a_r}$ est un simplexe inclus dans $\sigma_1 \cap \sigma_2$ et $\tau \cap \realisation{\Sigma_{k-1}} = \sigma$.
      \stopassertion

      \startdem
        Étant donné $H$ un hyperplan d'appui de $\sigma_1 \cap \sigma_2$ dans $\aff\p{\sigma_1 \cap \sigma_2}$, on écrit :
        \startformula
          H = \comprehension{x}{\prodScal{x}{u} = \theta}
          \quad\text{et}\quad
          \pourTout{x \in \p{\sigma_1 \cap \sigma_2}} \prodScal{x}{u} \leq \theta
        \stopformula
        avec $u$ un vecteur non nul et $\theta$ un réel. Un argument de dimension montre l'existence d'un sommet de $\sigma_1 \cap \sigma_2$ hors de $H$, d'où $\prodScal{b}{u} < \theta$ et $b \nin H$ (A CORRIGER).
      \stopdem
    \stopdem


    \startdiscussion[Subdivision cristalline]
      $\QED$
    \stopdiscussion
  \stopsection


\stopchapter

\stopcomponent
