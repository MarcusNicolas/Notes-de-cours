% Macros mathématiques de base
\startenvironment env_MacrosBase

  % TODO : changer la police de \cal

  % Écriture de preuves


  \define[2]\demEquivalence{
    \starttabulate[ |cmw(1cm)|p| ]
    \NC \p{\Rightarrow} \NC #1 \NC \NR
    \TB[halfline]
    \NC \p{\Leftarrow}  \NC #2 \NC \NR
    \stoptabulate}

  \define[2]\demDoubleInclusion{
    \starttabulate[ |cmw(1cm)|p| ]
    \NC \p{\subseteq} \NC #1 \NC \NR
    \TB[halfline]
    \NC \p{\supseteq}  \NC #2 \NC \NR
    \stoptabulate}


  % Quantificateurs

  \define[1]\ilExiste{\exists{#1},\;}
  \define[1]\pourTout{\forall{#1},\;}




  % MACROS A TRIER
  
  \definemathcommand[parties][nolop]{\mathfrak{P}}


  \define\p
    {\dosingleempty\dop}

  \def\dop[#1]#2{%
    \iffirstargument%
      #1(#2#1)%
    \else%
      \mathopen{}\left(#2\right)\mathclose{}%
    \fi%
  }



  \define[1]\abs{\mathopen{}\left\vert#1\right\vert\mathclose{}}


  \define[1]\classe
    {\mathopen{}\left\{#1\right\}\mathclose{}}
  
  \define[2]\comprehension
    {\classe{\:#1\;\,\middle|\,\;#2\:}}


  \define[2]\famille{\p{#1}_{#2}}

  \define[2]\segment{\left\[ #1, #2 \right\]}
  \define[2]\intervalle{\left\] #1, #2 \right\[}


  \define\iso{\simeq}


  \define\reels{\begingroup\mathbf{}R\endgroup}

\stopenvironment
